In the real world, we observe natural events unfold and sometimes do things outside of what one might predict. In the event of a flood, one would like to be able to accurately predict and visualize such events in order to better serve those in need. Or be able to understand the water run off effects in new development. However, when using only physical models, we are limited to not only the constraints of the built environment but rather more of the effects when simulating. It also limits the location of where said simulation can be as a water source needs to be present. When it comes to these physical models, it can not only be time consuming to create and setup but also require a lot of cleanups and resetting of the model. If there was a way to be able to solve these issues, we could see a major improvement for not only knowing how to better serve others in those events but also educate the public about these situations and making it more accessible and interactive for many to be able to experience anytime.
