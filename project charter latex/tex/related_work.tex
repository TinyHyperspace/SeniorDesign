There are several existing tools and technologies that address water simulation, flood modeling, or educational visualization, but each has significant limitations for the objectives of the AR Wetlands Watchers project.

The U.S. Army Corps of Engineers developed HEC-RAS, a highly regarded river analysis system capable of simulating complex water flow and floodplain scenarios \cite{HECRAS2024}. While extremely powerful, it requires domain expertise and is primarily designed for technical engineering applications, not for public education or interactive AR visualization.

Esri’s ArcGIS StoryMaps platform provides interactive web-based storytelling using geographic data \cite{ArcGIS2024}. While useful for presenting environmental data, it does not offer real-time physical interaction or immersive AR capabilities that engage non-technical audiences.

SimTable offers a sand-table-based simulation system combining physical models with projected simulations for wildfire and flood planning \cite{SimTable2023}. However, it involves specialized hardware, large physical space, and high costs, limiting its accessibility for widespread use in classrooms, public outreach, or portable community demonstrations.

FloodSim is a serious game designed to teach policy trade-offs in flood management \cite{FloodSim2023}. Although educational, it is screen-based and lacks real-world spatial interaction or physical engagement with models.

Finally, several AR environmental demos have been created using Microsoft HoloLens and other high-end AR devices \cite{HoloLens2023}. While technologically impressive, these systems often require expensive headsets that are impractical for widespread community deployment, especially in budget-constrained educational settings.

While these existing solutions offer valuable features, they each have significant limitations for our intended use case. Many require expensive specialized hardware, lack true real-time user interaction with physical spaces, or do not provide easily portable, low-cost options for schools, public outreach events, or disaster preparedness training. Our AR Wetlands Watchers project is designed specifically to address these gaps by offering:

\item Accessibility through common mobile devices.

\item Low-cost implementation using widely available AR platforms.

\item Hands-on interaction with simulated environments in real time.

\item Customizable educational content for both professionals and public audiences.

By focusing on these strengths, our solution will deliver a unique combination of technical capability and broad accessibility that existing systems do not fully achieve.