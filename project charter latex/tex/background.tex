Traditional methods of simulating flood events, water runoff, and environmental waterway management rely heavily on physical models and controlled environments. These models, while valuable, suffer from multiple drawbacks. They are often costly to build, limited in the locations where they can be deployed, time-consuming to reset between tests, and are bound by physical constraints that limit dynamic experimentation. Additionally, physical models require significant manual labor, materials, and maintenance, which often makes them impractical for frequent use or wide-scale educational purposes.



There is a growing opportunity to leverage advancements in Augmented Reality (AR) to overcome these limitations. With AR technologies now accessible through widely available consumer devices such as smartphones and tablets, simulations can be made more portable, interactive, and scalable. AR can allow users to visualize complex environmental systems in real time, experiment with different scenarios on demand, and educate diverse audiences without the need for complex physical setups

The AR Wetlands Watchers project is being developed with the aim of empowering communities, municipalities, educators, and businesses by providing a flexible and interactive AR-based tabletop simulation of water systems. Our sponsor is particularly interested in this solution for its potential to enhance public engagement, increase environmental awareness, and support disaster preparedness through interactive educational experiences. This project serves as both a technical exploration and a public outreach tool, fostering better understanding of how human activities influence water systems and how communities can prepare for extreme environmental events such as floods and contamination.

